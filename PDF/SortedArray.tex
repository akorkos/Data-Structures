\chapter{Ταξινομημένος πίνακας}

\section{Εισαγωγή}

Στην δομή αυτή θα γίνει η χρήση ταξινόμησης (ταξινόμηση με εισαγωγή), δηλαδή κάθε φορά που εισάγεται κάποιο στοιχείο θα εισάγεται στην θέση αυτή έτσι ώστε να υπάρχει λεξικογραφική διάταξη ανάμεσα στις λέξεις του κειμένου. Επιπλέον για την επιτάχυνση των αποτελεσμάτων, όπου απαιτείται αναζήτηση θα χρησιμοποιηθεί η δυαδική αντί της σειριακής που χρησιμοποιείται στον αταξινόμητο. 

\section{Λειτουργίες}

\subsection{Εισαγωγή}

Για την εισαγωγή στο ταξινομημένο πίνακα θα γίνει η χρήση της δυαδικής αναζήτησης και της ταξινόμησης με εισαγωγή. Αρχικά γίνεται η κλήση μια παραλλαγής της δυαδικής αναζήτησης της οποίας το αποτέλεσμα αποθηκεύετε σε μια μεταβλητή. Με παραλλαγή εννοείται πως όταν το στοιχείο υπάρχει ήδη στον πίνακα, η μεταβλητή found που περνιέται με στην συνάρτηση με αναφορά γίνεται true και επιστρέφεται η θέση του στοιχείου μέσα στον πίνακα, αντίθετα όταν το στοιχείο απουσιάζει η δυαδική αναζήτηση βρίσκει την θέση στην οποία πρέπει να τοποθετηθεί το στοιχείο για να διατηρηθεί η ταξινόμηση και η μεταβλητή found γίνεται false. Έπειτα εξετάζεται από την συνάρτηση εισαγωγής η τιμη της found, εάν είναι true αυξάνεται η τιμή της μεταβλητής appearances του Element (struct) που βρίσκεται μέσα στον πίνακα κατά ένα (1) διαφορετικά, κάθε στοιχείο το οποίο βρίσκεται δεξιά από τη θέση που θα πρέπει να εισαχθεί το νέο στοιχείο, μετακινείται κατά μια (1) θέση προς δεξιά και έπειτα το στοιχείο εισάγεται στην σωστή θέση χωρίς να διαταραχθεί η ταξινόμηση.

\subsection{Αναζήτηση}

Για την αναζήτηση των στοιχείων μέσα από τον πίνακα θα χρησιμοποιηθεί η δυαδική αναζήτηση έτσι ώστε να γίνει εκμετάλλευση της ταξινόμησης. Αφού βρεθεί η θέση του στοιχείου μέσα στον πίνακα επιστρέφεται ο αριθμός των εμφανίσεων της συγκεκριμένης λέξης αλλά και η τιμή true (επιτυχής αναζήτηση), σε κάθε άλλη περίπτωση επιστρέφεται η τιμή false. 

\subsection{Διαγραφή}

Όπως στον αταξινόμητο πίνακα, έτσι και εδώ γίνεται κλήση της συνάρτησης (δυαδικής) αναζήτησης ώστε να βρεθεί η θέση της λέξης που πρόκειται να διαγραφεί. Σε περίπτωση που η λέξη εμφανίζεται περισσότερες από μια φορές, το πεδίο του Element (struct) που αποθηκεύει τον αριθμό των εμφανίσεων της λέξης. Εάν η τιμή του πεδίου πριν την κλήση της συνάρτησης είναι ένα (1) τότε, αφού κληθεί αφαιρείται η λέξη από τον πίνακα, τα στοιχεία μετακινούνται κατά μια θέση προς τα αριστερά και το μέγεθος του πίνακα μειώνεται κατά ένα. Παράλληλα όταν η λέξη είναι παρόν στο πίνακα πέρα από την παραπάνω διαδικασία επιστρέφεται και η τιμή true ενώ όταν δεν υπάρχει η λέξη για διαγράφει η τιμή false. 
