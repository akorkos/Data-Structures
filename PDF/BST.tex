\chapter{Απλό δυαδικό δένδρο αναζήτησης}

\section{Εισαγωγή}

Για την αποθήκευση των λέξεων μέσα στο δυαδικό δένδρο χρησιμοποιείται ένα struct (node), η δομή του struct είναι η εξής: χρησιμοποιείται η μεταβλητή word τύπου string για την αποθήκευση της λέξης, μια μεταβλητή appearances τύπου int για την αποθήκευση των εμφανίσεων της συγκεκριμένης λέξης και υπάρχουν τρεις μεταβλητές (parent, left, right) τύπου node (struct) για την αποθήκευση του προγόνου (πατέρα) κόμβου, την αποθήκευση του αριστερού απογόνου (παιδιού) και του δεξιού απογόνου (παιδιού) αντίστοιχα. 
Για την δημιουργία λοιπόν του δένδρου, θα χρησιμοποιηθεί η root τύπου node που θα αναπαριστά την ρίζα του δένδρου (η τιμή της root κατά την κατασκευή της δομής θέτεται σε nullptr).


\section{Λειτουργίες}

\subsection{Eισαγωγή (insert)}

Για την εισαγωγή των λέξεων στο πίνακα από τον χρήστη καλείται η μέθοδος insert (public), η οποία εξετάζει αρχικά την ρίζα (root) του δένδρου δεν έχει πάρει κάποια λέξη ως τιμή (root == nullptr). Εάν είναι άδειος ο κόμβος τότε η λέξη τοποθετείται στην ρίζα και γίνεται αρχικοποίηση των μεταβλητών του node (appearances = 1, parent = nullptr, right = nullptr, left = nullptr), σε κάθε άλλη περίπτωση καλείται η insert (private), σε αυτή την μέθοδο εξετάζεται εάν η λέξη είναι ίδια με αυτή που έχει ήδη ο κόμβος εάν ναι η μεταβλητή appearances αυξάνεται κατά ένα (1), εάν όχι εξετάζεται εάν είναι λεξικογραφικά μεγαλύτερη από την λέξη που υπάρχει ήδη στον κόμβο δημιουργείται ένας νέος κόμβος δεξιά του κόμβου που εξετάζεται διαφορετικά, (δηλ. είναι λεξικογραφικά μικρότερη) δημιουργείται ένας νέος κόμβος που τοποθετείτε η λέξη στα αριστερά του κόμβου που εξετάζεται.

\subsection{Διαγραφή (remove)}

\subsection{Aναζήτηση (search)}

\section{Διάσχιση}

Η λειτουργίες τις διάσχισης του δένδρου υλοποιούνται όλες αναδρομικά.

\subsection{Ενδοδιατεταγμένη (inorder)}

Για κάθε κόμβο, επισκεπτόμαστε πρώτα τους κόμβους του αριστερού του υποδένδρου, έπειτα τον ίδιο τον κόμβο και στη συνέχεια τους κόμβους του δεξιού του υποδένδρου.

\subsection{Μεταδιατεταγμένη (postorder)}

Για κάθε κόμβο, επισκεπτόμαστεπρώτα τους κόμβους του αριστερού του υποδένδρου, έπειτα τους κόμβους του δεξιού του υποδένδρουκαι στη συνέχεια τον ίδιο τον κόμβο.

\subsection{Προδιατεταγμένη (preorder)}

Για κάθε κόμβο, επισκεπτόμαστεπρώτα τον ίδιο τον κόμβο, έπειτα τους κόμβους του αριστερού του υποδένδρουκαι στη συνέχεια τους κόμβους του δεξιού του υποδένδρου.
